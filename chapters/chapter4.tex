\chapter{Future Scope}
Not all cells in the Euler transformation $\hK$ are guaranteed to be convex, even when all cells in $K$ are (see \cref{fig:disjholes}).
We could triangulate the non-convex cells so that all cells in $\hK$ are convex.
But could we do so while maintaining even degrees for all vertices?
A related problem is that of finding a triangulation (rather than a cell complex) of a given domain that minimizes the number of odd-degree vertices.
The total Euclidean length of edges is $\tilde{K}$ is going to be at least double compared to that in the original complex in $K$.
Hence it is better to start with a sparse input complex $K$ (i.e., with a smaller total Euclidean length of edges).
We have described a complete framework for continuous tool path planning in layer-by-layer 3D printing.
The clipping step will be bottlenecked by the computation of intersection of the Euler transformed complex with each polygon in each layer.
We have generalized the Euler transformation defined to allow combinatorial changes when computing mitered offsets of cells.
What about allowing topological changes?
It appears applying the generalized Euler transformation should be able to generate an Euler complex even when topological changes are allowed.
But there might be some new geometric challenges generated in this process, which would have to be taken care of.
We will address this question in future work.
Another promising generalization of our approach would be to \emph{non-planar} 3D printing.
Many of our results should generalize to the non-planar realm as long as underlying support is guaranteed by the design.

We have developed a decomposition approach to solve large instances of optimized path planning problem in 3d printing where each sub-polygon is guaranteed to have a dual graph with a feasible tool path. 
Our framework guarantees that discontinuities in the tool path, if any, are located only at the boundary of the original (input) polygon.
Further, we can change the Hilbert ordering of the cells by changing enter and exit corner vertices of the initial cell in the quadtree.
We can also create various decompositions for the same IOP by changing parameters $\delta$ and $\Delta$.

The edge weights in our graphical framework could model multiple quality factors including turn costs, edge overlap across adjacent layers, tool path length, and others.
Our mechanical testing has shown that changing the extent of overlap across layers could impact the mechanical strength of the printed object.
Another potential application of our framework is the optimization of internal microstructure and thermal management by choosing appropriately defined edge weights derived from physical models and/or experiments, which in turn could result in increased strength of the printed objects.

For the Buddha and Bunny together, our framework solved more than 10,000 IP subproblems across more than 500 layers.
These IP instances could be solved independently, and hence in an embarrassingly parallel fashion.
Alternatively, our framework allows the use of approximation algorithms or heuristics to solve the subproblems instead of using IP \cite{ApBiChCo2007}.
We could also reuse optimal solutions for cells that reoccur across multiple layers.
Using uniform edge weights, and varying the relative importance parameter $\alpha$ (Equation \ref{eq:IPobj}), we could obtain fractal-like patterns for the toolpath.  

If we solve the full IP model (including subtour constraint \ref{eq:IPsubtr}), any discontinuities in the tool path are located at the boundary of the original polygon. 
This could be especially of concern when the polygon is relatively thin as compared to size of extruder.
We could consider reducing the extruder size to handle such situations, or consider alternative methods (e.g., spiral or zigzag patterns).
%\textcolor{blue}{With} relaxed IP model and associated heuristics, we cannot guarantee a single continuous path for each cell.
In extreme cases where the polygon in a given layer has many sharp curvature regions, we could have many small sized sub-polygons near the boundary.
This setting could create several discontinuities in the tool path at the boundary.
We will explore methods to handle such extreme cases in our future work.

Our mechanical testing experiments (Section \ref{sec:mechtest}), while preliminary, already demonstrate that the amount of edge overlap across adjacent layers could significantly affect the strength of the print.
We plan to employ the SFCDecomp framework to study in detail the effects of tool path design as well as edge overlap across layers on various mechanical properties.